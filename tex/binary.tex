\section{Binary terms, algebraic structure}

In the following proposition we collect what we know until now. 

\begin{proposition}
The following are true: 
\begin{enumerate}
\item[-] for any $a \in \Upsilon$  $\displaystyle (a - a)^{e} x \, \longleftrightarrow \, 0^{e} x$
\item[-] for any $a \in \Upsilon$ and $b$ binary term $\displaystyle \left(a - \left(a - b\right)\right)^{e} x \, \longleftrightarrow \, b^{e} x$
\item[-] for any $a \in \Upsilon$  $\displaystyle \left(a 0 \right)^{e} x \, \longleftrightarrow \, 0^{e} x$
\item[-] for any binary term $b$  $\displaystyle \left(b 1 \right)^{e} x \, \longleftrightarrow \, b^{e} x$
\item[-] for any binary term $b$  $\displaystyle \left(1 b \right)^{e} x \, \longleftrightarrow \, b^{e} x$
\item[-] for any binary term $b$  $\displaystyle \left(0 b \right)^{e} x \, \longleftrightarrow \, 0^{e} x$
\item[-] if $b$ is a binary term such that $\displaystyle  \left(b 0 \right)^{e} x \, \longleftrightarrow \, 0^{e} x$ then for any $a \in \Upsilon$ $\displaystyle \left(\left(a-b\right) 0 \right)^{e} x \, \longleftrightarrow \, 0^{e} x$
\item[-] if $b, c$ are binary term such that $\displaystyle  \left(b 0 \right)^{e} x \, \longleftrightarrow \, 0^{e} x$,  $\displaystyle  \left(c 0 \right)^{e} x \, \longleftrightarrow \, 0^{e} x$ then $\displaystyle  \left(\left(b c\right) 0 \right)^{e} x \, \longleftrightarrow \, 0^{e} x$ 
\item[-] for any $a,b \in \Upsilon$ $\displaystyle \left(a b \right)^{e} x \, \longleftrightarrow \, \left(b a  \right)^{e} x$
\item[-] for any $a,b \in \Upsilon$ $\displaystyle \left(\bar{a} \bar{b} \right)^{e} x \, \longleftrightarrow \, \left(\bar{b} \bar{a}  \right)^{e} x$
\item[-] for any $a,b \in \Upsilon$ $\displaystyle \left(\bar{a} b \right)^{e} x \, \longleftrightarrow \, \left(b \bar{a}  \right)^{e} x$
\item[-] for any $a \in \Upsilon$ $\displaystyle \left(\bar{a} a \right)^{e} x \, \longleftrightarrow \, 1^{e} x  \, \longleftrightarrow \, \left(a \bar{a}  \right)^{e} x$
\end{enumerate}

\end{proposition}

\begin{definition}
$\Gamma$ is the set of binary terms generated by: 
\begin{enumerate}
\item[-] for any $a \in \Upsilon$, $\displaystyle a^{e} x$, $\displaystyle \bar{a}^{e} x$ are in $\Gamma$
\item[-] $\displaystyle 1 \in \Gamma$
\item[-] if $a, b \in \Gamma$ then $ab \in \Gamma$.
\end{enumerate}
We extend the notation $\displaystyle \bar{\cdot}$ to all of $\Gamma$ by: if $a, b \in \Gamma$ then $\displaystyle \bar{ab} \, = \, \bar{b} \bar{a}$ and $\bar{\bar{a}} \, = \, a$. 

$\displaystyle \bar{\Gamma} \, = \, \Gamma \cup \left\{ 0 \right\}$. 

$B\Upsilon$ is the set of binary terms generated by: 
\begin{enumerate}
\item[-] for any $a \in \Upsilon$, $\displaystyle a^{e} x$, $\displaystyle \bar{a}^{e} x$ are in $B\Upsilon$
\item[-] $\displaystyle 1 \in B\Upsilon$ and $\displaystyle 0 \in B\Upsilon$
\item[-] if $a, b \in B\Upsilon$ then $ab \in B\Upsilon$
\item[-] if $a \in \Upsilon$ and $b \in b\Upsilon$ then $\displaystyle a-b \in B\Upsilon$ 
\end{enumerate}
\end{definition}


\begin{proposition}
The set $\displaystyle \bar{\Gamma}$ is a commutative monoid with the operation of multiplication, neutral element $1$ and equality relation $\longleftrightarrow$. The set $\Gamma$ is a commutative group, subset of the monoid  $\displaystyle \bar{\Gamma}$, with the inverse unary operation generated by $\displaystyle a^{-1} = \bar{a} \, \bar{a}^{-1} = a$, for any $a \in \Upsilon$, and $\displaystyle 1^{-1} = 1$. 

The set $B\Upsilon$ is a noncommutative monoid with multiplication, neutral element $1$ and equality relation $\longleftrightarrow$. Moreover, for any $b, c \in B\Upsilon$ and for any $a \in \Upsilon$
\begin{enumerate}
\item[-] $a-b \in B\Upsilon$
\item[-] $0b \longleftrightarrow 0 \longleftrightarrow b0$ (i.e. $0$ is an absolute of the monoid $B\Upsilon$)
\item[-] $a-(a-b) \longleftrightarrow b$
\item[-] $a-0 \longleftrightarrow a$
\item[-] $\displaystyle a-1 \longleftrightarrow inv_{a}$ (see Definition \ref{aterms})
\end{enumerate}
\end{proposition}
