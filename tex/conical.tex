\section{Conical groups}


An algebraic conical group $\displaystyle (G, \oplus, \ominus, e, Y, \bar{\cdot}, 1, m)$ is: 
\begin{enumerate}
\item[-] a group $G$  with the operation $(x,y) \mapsto x \oplus y$  inverse $x \mapsto \ominus x$ and difference operation $\displaystyle (x,y) \mapsto \ominus x \oplus y \, = \, \left( \ominus x \right) \oplus y$ and neutral element $e$. The axioms are: for any $x, y, z \in G$ 
\begin{equation}
\begin{array}{rclr}
\left(x \oplus y \right) \oplus z & = & x \oplus \left( y \oplus z \right) & \mbox{(associativity)} \\ 
 e \oplus x  & = & x  & \mbox{(left-neutral element)} \\ 
 x \oplus e  & = & x  &  \mbox{(right-neutral element)} \\
\ominus \left( \ominus x \right) & = & x  & \mbox{(idempotent-inv)} \\
 \left(\ominus x\right) \oplus x  & = & e  & \mbox{(left-inverse)} \\
 x \oplus \left(\ominus x \right)  & = & e  & \mbox{(right-inverse)}
\end{array}
\label{gpluseqs}
\end{equation}
\item[-] a commutative group $Y$ with operation $(a,b) \mapsto ab$, inverse $\displaystyle a \mapsto \bar{a}$ and neutral element $1$. The axioms are: for any $a, b, c \in Y$
\begin{equation}
\begin{array}{rclr}
\left(a b \right) c & = & a \left( b c \right) & \mbox{($Y$-associativity)} \\ 
 1 a  & = & a  & \mbox{($Y$-left-neutral element)} \\ 
 a 1  & = & a  & \mbox{($Y$-right-neutral element)} \\ 
 \bar{\bar{a}} & = & a  & \mbox{($Y$-idempotent-inv)} \\ 
 \bar{a} a  & = & 1  & \mbox{($Y$-left-inverse)} \\ 
 a \bar{a}  & = & 1  & \mbox{($Y$-right-inverse)} \\ 
 a b  & = & b a  & \mbox{($Y$-commutative)}
\end{array}
\label{ycomm}
\end{equation}
\item[-] a multiplication operation $\displaystyle (a,x) \in Y \times G \mapsto m(a,x) = ax \in G$, which satisfies the axioms: for any $a, b \in Y$ and for any $x, y \in G$ 
\begin{equation}
\begin{array}{rclr}
a\left(b x \right) & = & \left( a b \right) x & \mbox{(left-action)} \\ 
1 x  & = &  x & \mbox{(left-action-neutral)} \\ 
a\left(x \oplus y \right) & = &  \left(a x\right) \oplus \left(a y\right) & \mbox{(distributive)} \\ 
a e & = & e & \mbox{(fixed-point)}
\end{array}
\label{multax}
\end{equation}
\end{enumerate}

\begin{definition}
A topological conical group is an algebraic conical group, where $G$ and $Y$ are topological groups, all operations are continuous, such that: 
\begin{enumerate}
\item[-] $Y$ has an absolute $0$, i.e. we can add an element $0$ to $Y$ and we can extend by continuity the operation on $Y$ by 
\begin{equation}
a 0 \, = 0 a \, = \, 0
\label{yabsolute}
\end{equation}
in the sense that for any $b \in Y$ we have $\displaystyle \lim_{a \rightarrow 0} ab = 0$. 
\item[-] we can extend by continuity the multiplication of elements in $G$ with the absolute $0$ by: 
\begin{equation}
m(0,x) \, = \, 0 x \, = \, e
\label{g0x}
\end{equation}
where the element $e$ from the right hand side of (\ref{g0x}) is the neutral element of $G$, in the sense that 
 for any $x \in G$ we have $\displaystyle \lim_{a \rightarrow 0} m(a,x) \, = \, e$.
\end{enumerate}
\end{definition}

An algebraic conical group is different from a vector space or even from a left module. Indeed: 
\begin{enumerate}
\item[-] we think about the elements of $Y$ as scalars, but $Y$ is not a ring, it is only a commutative group. There is no addition of scalars.
\item[-] we think about the elements of $G$ as vectors, but the addition of vectors is not commutative. 
\item[-] because there is no addition of scalars, there is no distributivity of scalar multiplication with respect to scalars addition.
\end{enumerate}
However, trivially if $Y$ is a field and if $G$ is a $Y$-vector space, then we obtain an algebraic conical group which is commutative and such that the distributivity of scalar multiplication with respect to scalars addition is satisfied. 

There are many other examples of conical groups. 

\begin{definition}
Let $G$ be a group with neutral element e,  endowed with a group automorphism $\phi: G \rightarrow G$. Let $\displaystyle Y = (\mathbb{Z}, + , 0)$. Define the multiplication 
$$ m(a,x) \, = \, \phi^{a}(x) $$
where $\displaystyle \phi^{a} \, = \, \phi \circ ... \circ \phi$ $a$ times if $a >0$, $\displaystyle \phi^{0} \, = \, id_{G}$ and  $\displaystyle \phi^{a} \, = \, \phi^{-1} \circ ... \circ \phi^{-1}$  $-a$ times if $a < 0$. We get a structure of an algebraic conical group. 

If moreover the group $G$ is a topological group, the automorphism $\phi$ is continuous, with continuous inverse and for any $x \in G$ we have $\displaystyle \lim_{n \rightarrow \infty} m(n,x) \, = \, e$ then we obtain the definition of a contractible group \cite{siebert}.

\end{definition}

Contractible groups are therefore topological conical groups, with $\displaystyle Y = (\mathbb{Z}, + )$ and with absolute $\infty$. 

A topological real vector space is a topological conical group, with $\displaystyle Y = \mathbb{R}^{*}$ and with absolute $0 \in \mathbb{R}$. However remark that we still don't use the additions of real scalars. 

Carnot groups are the inspiration for topological conical groups. 
